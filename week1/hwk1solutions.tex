 
\documentclass[12pt]{article}
 %author David Dobor
\usepackage[margin=1in]{geometry} 
\usepackage{amsmath,amsthm,amssymb}
 
\newcommand{\N}{\mathbb{N}}
\newcommand{\Z}{\mathbb{Z}}


\newcommand{\blditA}{\textbf{\textit{A}}}
\newcommand{\blditB}{\textbf{\textit{B}}}
\newcommand{\blditC}{\textbf{\textit{C}}}
\newcommand{\blditP}{\textbf{\textit{P}}}
\newcommand{\blditQ}{\textbf{\textit{Q}}}
\newcommand{\bldI}{\textbf{I}}
\newcommand{\blditX}{\textbf{\textit{X}}}
 
\newenvironment{theorem}[2][Theorem]{\begin{trivlist}
\item[\hskip \labelsep {\bfseries #1}\hskip \labelsep {\bfseries #2.}]}{\end{trivlist}}
\newenvironment{lemma}[2][Lemma]{\begin{trivlist}
\item[\hskip \labelsep {\bfseries #1}\hskip \labelsep {\bfseries #2.}]}{\end{trivlist}}
\newenvironment{exercise}[2][Exercise]{\begin{trivlist}
\item[\hskip \labelsep {\bfseries #1}\hskip \labelsep {\bfseries #2.}]}{\end{trivlist}}
\newenvironment{problem}[2][Problem]{\begin{trivlist}
\item[\hskip \labelsep {\bfseries #1}\hskip \labelsep {\bfseries #2.}]}{\end{trivlist}}
\newenvironment{question}[2][Question]{\begin{trivlist}
\item[\hskip \labelsep {\bfseries #1}\hskip \labelsep {\bfseries #2.}]}{\end{trivlist}}
\newenvironment{answer}[2][Answer]{\begin{trivlist}
\item[\hskip \labelsep {\bfseries #1}\hskip \labelsep {\bfseries #2.}]}{\end{trivlist}}

\begin{document}
 

 
\title{Homework 1}%replace X with the appropriate number
\author{David Dobor, Andrew Schneider,  Abdulsalam Hdadi\\ %replace with your name
Stat 8003} %if necessary, replace with your course title
 
\maketitle
 
 %%%%%%%% Question 1 %%%%%%%%%
 
\begin{question}{1.1} %You can use theorem, exercise, problem, or question here.  Modify 1.1to be whatever number you are proving
Let \blditA $ $ and \blditB $ $ be two matrices defined as:

\begin{equation*}
A = \begin{pmatrix}
  1 & 2 & 3\\
  2 & 4 & 6\\
  3 & 6 & 9
\end{pmatrix},
B = \begin{pmatrix}
  2 & 0\\
  1 & 3\\
 -2 & 1
\end{pmatrix}
\end{equation*}

Calculate:

\begin{itemize}
  \item $\blditA \blditB$
  \item $\blditB^T \blditA $
\end{itemize}

Use R to check your calculation.

 \end{question}
 
 
 \textbf{\emph{Answer:} }
 
`By hand` calculation yields:
\begin{equation*}
\blditC  = \blditA \blditB = 
\begin{pmatrix}
  -2 & 9\\
  -4 & 18\\
  -6 & 27
\end{pmatrix}
\end{equation*}
 For example, the entry in the third row and second column of \blditC, $c_{3,2}$, is calculated as:
 
 \begin{equation*}
 c_{3,2} = a_{3,1}\times b_{1,2} +  a_{3,2}\times b_{2,2}  +  a_{3,3}\times b_{3,2} = 3\times 0 + 6\times 3 + 9\times 1 = 27
 \end{equation*}
 
 Similarly,

\begin{equation*} 
\blditB^T \blditA= 
 \begin{pmatrix}
  2 & 1 & -2\\
  0 & 3 & 1\\
\end{pmatrix}
 \begin{pmatrix}
  1 & 2 & 3\\
  2 & 4 & 6\\
  3 & 6 & 9
\end{pmatrix} = 
 \begin{pmatrix}
   -2 & -4 & -6\\
    9  & 18 & 27
\end{pmatrix}
 \end{equation*}
 whose entry in the first row second column is calculated as:
  \begin{equation*}
 2\times 2 + 1\times 4 + (-2)\times 6 = -4
 \end{equation*}
 
 The following {\tt R} code can be used to verify these results:
 
 \begin{verbatim}
	# Solution to Question1, Homework 1

	#given matrix A:
	col1 <- c(1,2,3)
	A <- cbind(col1, 2*col1, 3*col1)
	#and matrix B:
	B <- matrix((c(2, 1, -2, 0, 3, 1)), ncol = 2, nrow = 3)

	#compute the following prodcts:
	A %*% B
	t(B) %*% A
\end{verbatim}

\bigskip
\bigskip
%%%%%%%% Question 2 %%%%%%%%%

\begin{question}{1.2} 
If $\blditA$ is invertible, prove that $det(\blditA^{-1} ) = (det(\blditA))^{-1}$
\end{question}
\begin{proof}
$\textbf{\textit{A}} $ being invertible, consider the product $\blditA \blditA^{-1} = \bldI$. By property (b) on page 17 of the lecture notes, we have:
\begin{equation*} 
1 = det(\bldI) = 
det(\blditA \blditA^{-1}) = det(\blditA) det(\blditA^{-1})
\end{equation*}
from which
\begin{equation*}
det(\blditA^{-1}) = \frac{1}{det(\blditA)}
\end{equation*}
\end{proof}


\bigskip
\bigskip
%%%%%%%% Question 3 %%%%%%%%%

\begin{question}{1. 3. a} 
If matrix \blditP is idempotent, then $\blditQ = \bldI - \blditP$ is also idempotent.
\end{question}

\begin{proof}
Since
\begin{equation*}
\blditQ^2 = (\bldI - \blditP)^2 = \bldI^2 - \bldI \blditP - \blditP \bldI + \blditP^2 = \bldI - \blditP - \blditP + \blditP = \bldI - \blditP 
\end{equation*}
where the next-to-last equality follows because both \bldI and \blditP are idempotent. Thus
\begin{equation*}
\blditQ^2 = \blditQ \text{, as claimed}
\end{equation*}
\end{proof}



\bigskip
\begin{question}{1. 3. b} 
If \blditX is an $n \times m$ matrix with rank $m$, show that the following matrix \blditP is idempotent
\begin{equation*}
\blditP = \blditX (\blditX^T \blditX)^{-1}\blditX^T
\end{equation*}

\end{question}

\begin{proof}
We have:
\begin{align*}
\blditP^2 &= (\blditX (\blditX^T \blditX)^{-1}\blditX^T)^2 \\
 & = \blditX (\blditX^T \blditX)^{-1}\blditX^T \blditX (\blditX^T \blditX)^{-1}\blditX^T \\
 & = \blditX (\blditX^T \blditX)^{-1}(\blditX^T \blditX) (\blditX^T \blditX)^{-1}\blditX^T 
\end{align*}
Noting that in the middle term of this we've got:
\begin{equation*}
(\blditX^T \blditX) (\blditX^T \blditX)^{-1} = \bldI
\end{equation*}
We end up with:
\begin{equation*}
\blditP^2 = \blditX (\blditX^T \blditX)^{-1}\blditX^T = \blditP
\end{equation*}
\end{proof}

\bigskip
\bigskip
%%%%%%%% Question 4 %%%%%%%%%
\begin{question}{1. 4} 
Given matrix $\blditA =  \begin{pmatrix}
  2 & -1 & 0\\
  -1 & 2 & -1\\
  0 & -1 & 2
\end{pmatrix} $. Is \blditA $ $ positive-definite? Prove it or disprove it.
\end{question}

\textbf{\emph{Answer:} }

Yes, it is positive-definite:
\begin{proof}
Blah, blah, blah.  
\end{proof}


\bigskip
\bigskip
%%%%%%%% Question 5 %%%%%%%%%
\begin{question}{1. 5} 
The Gamma $\Gamma(\alpha)$ function is defined as:
\begin{equation*}
\Gamma(\alpha) = \int_0^\infty x^{\alpha - 1}\mathrm{e}^{-x}\,\mathrm{d}x
\end{equation*}

\begin{enumerate}
  \item Prove that $\Gamma(\alpha + 1) = \alpha\Gamma(\alpha)$
  \item Calculate $\Gamma(n)$ where $n$ is a positive integer
  \item Calculate $\int_0^\infty x^{-\alpha - 1}\mathrm{e}^{-\frac{\beta}{x}}\,\mathrm{d}x$, express your result using Gamma function
\end{enumerate}

\end{question}

\begin{proof}
Blah, blah, blah.  
\end{proof}



\end{document}
%Note 1: The * tells LaTeX not to number the lines.  If you remove the *, be sure to remove it below, too.
%Note 2: Inside the align environment, you do not want to use $-signs.  The reason for this is that this is already a math environment. This is why we have to include \text{} around any text inside the align environment.

%\begin{proof}
%Blah, blah, blah.  I'm so smart.
%\end{proof}

