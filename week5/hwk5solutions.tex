 
\documentclass[12pt]{article}
%author David Dobor
\usepackage[margin=1in]{geometry} 
\usepackage{amsmath,amsthm,amssymb}
 \usepackage{graphicx}
 \usepackage{multirow}
\usepackage[scaled]{helvet}
\usepackage{hyperref}
\usepackage[usenames,dvipsnames,svgnames,table]{xcolor}
\usepackage[T1]{fontenc}
\usepackage{palatino}
\usepackage{enumerate}
%\renewcommand*\familydefault{\sfdefault} %% Only if the base font of the document is to be sans serif

\newcommand{\N}{\mathbb{N}}
\newcommand{\Z}{\mathbb{Z}}


\newcommand{\blditA}{\textbf{\textit{A}}}
\newcommand{\blditB}{\textbf{\textit{B}}}
\newcommand{\blditC}{\textbf{\textit{C}}}
\newcommand{\blditP}{\textbf{\textit{P}}}
\newcommand{\blditQ}{\textbf{\textit{Q}}}
\newcommand{\bldI}{\textbf{I}}
\newcommand{\blditX}{\textbf{\textit{X}}}
 
\newenvironment{theorem}[2][Theorem]{\begin{trivlist}
\item[\hskip \labelsep {\bfseries #1}\hskip \labelsep {\bfseries #2.}]}{\end{trivlist}}
\newenvironment{lemma}[2][Lemma]{\begin{trivlist}
\item[\hskip \labelsep {\bfseries #1}\hskip \labelsep {\bfseries #2.}]}{\end{trivlist}}
\newenvironment{exercise}[2][Exercise]{\begin{trivlist}
\item[\hskip \labelsep {\bfseries #1}\hskip \labelsep {\bfseries #2.}]}{\end{trivlist}}
\newenvironment{problem}[2][Problem]{\begin{trivlist}
\item[\hskip \labelsep {\bfseries #1}\hskip \labelsep {\bfseries #2.}]}{\end{trivlist}}
\newenvironment{question}[2][Question]{\begin{trivlist}
\item[\hskip \labelsep {\bfseries #1}\hskip \labelsep {\bfseries #2.}]}{\end{trivlist}}
\newenvironment{answer}[2][Answer]{\begin{trivlist}
\item[\hskip \labelsep {\bfseries #1}\hskip \labelsep {\bfseries #2.}]}{\end{trivlist}}

\begin{document}
 \renewcommand{\arraystretch}{1.3}

 
\title{Stat 8003, Homework 5}%replace X with the appropriate number
\author{Group G: \ \ \texttt{sample( c( "David" , "Andrew",  "Salam" ))}
\\ %replace with your name
} %if necessary, replace with your course title
 
\maketitle
 
 %%%%%%%% Question 1 %%%%%%%%%
 \begin{question}{4.1}  
 
 Consider a simulated dataset. Assume that the data $x_1, x_2, \cdots , x_n$ follows the following distribution:
 $$
 x_i \sim f(x_i) = \pi_0 f_0(x_i) + \pi_1 f_1(x_i)
 $$
 where $f_0(x_i) = 1(0 \leq x_i \leq 1)$ is the density function of the uniform and $f_1(x_i) = \beta (1 - x)^{\beta - 1} $ is the density function of $Beta(1, \beta)$. The group information can be treated as a
missing value and is denoted as $z_i$. Let $y_i = (x_i, z_i)$ be the complete data.	
 \begin{enumerate}[(a)]
\item Derive the complete likelihood function;
\item Using the EM algorithm to derive the estimator for $\pi_0$ and $\beta$;
\item Apply your method to the data set, estimate $\pi_0$ and $\beta$ and the calculate $fdr_i = P(Z_i = 0 \mid x_i)$. (This score is called the local fdr score.)
\item Classify $x_i$ to the first group if $fdr_i(x_i) > 0.5$. Compare your classification with the actual
group information, what is the total number of falsely classified data?
\end{enumerate}
\end{question} 


  \textbf{\color{TealBlue}\emph{Answer:} } 
\bigskip
\bigskip
 %%%%%%%% Question 2 %%%%%%%%%
 \begin{question}{4.2} (Continued from Problem 1.)  It is known that the local fdr score can be written as 
$$
fdr_i(x_i) = \frac{\pi_0 f_0(x_i)} {f(x_i)}
$$
where $f(x_i)$ is the marginal density of $x_i$. Assume that $\pi = 0.7$.
 \begin{enumerate}[(a)]
\item Estimate $f(x_i)$ by using the kernel density estimation with Gaussian kernel and Silverman's $h$;
\item Estimate the local $fdr$ score;
\item  Using the same rule as in 1(d), calculate the total number of falsely classified data;
\item Choose the bandwidth using the maximum likelihood cross validation, repeat problem (a-c),
what is the total number of falsely classified data?
\item Which method works the best in terms of having the smallest classification error?
\end{enumerate}
\end{question} 


  \textbf{\color{TealBlue}\emph{Answer:} } 
 
 
 
 
 
 
 %%%%%%%% Question 3 %%%%%%%%%

%\begin{question}{4.3} 
% In the shuttle example, let $X_i$ denote the number of damaged o-rings and $t_i$
%the temperature, where $i = 1, 2, \dots, n$. Assume the model as
%$$
%\begin{cases} 
%       X_i \mid p_i \sim \text{Binom}(2, p_i) \\
%       p_i = \mathrm{e}^{(\beta_0 + \beta_1 t_i)} /  ( 1 +  {\mathrm{e}^{(\beta_0 + \beta_1 t_i)})}
%\end{cases}
%$$
%\end{question}
% \begin{enumerate}[(a)]
%\item Derive the log-likelihood function;
%
%Since
%\begin{align*}
%f(x_i \mid p_i) &= {2 \choose x_i} \left(\frac{ exp(\beta_0 + \beta_1 t_i) } { 1 +  exp(\beta_0 + \beta_1 t_i)} \right)^{x_i}     \left( \frac{1} { ( 1 +  exp(\beta_0 + \beta_1 t_i) } \right)^{2 - x_i} \\
%&=  {2 \choose x_i}  \frac{ exp(\beta_0 + \beta_1 t_i)^{x_i} } { (1 +  exp(\beta_0 + \beta_1 t_i) )^2} \\
%\end{align*}
%

\end{document}


