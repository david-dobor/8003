 
\documentclass[12pt]{article}
%author David Dobor
\usepackage[margin=1in]{geometry} 
\usepackage{amsmath,amsthm,amssymb}
 \usepackage{graphicx}
 \usepackage{multirow}
\usepackage[scaled]{helvet}
\usepackage{hyperref}
\usepackage[usenames,dvipsnames,svgnames,table]{xcolor}
\usepackage[T1]{fontenc}
\usepackage{palatino}
\usepackage{enumerate}
%\renewcommand*\familydefault{\sfdefault} %% Only if the base font of the document is to be sans serif

\newcommand{\N}{\mathbb{N}}
\newcommand{\Z}{\mathbb{Z}}


\newcommand{\blditA}{\textbf{\textit{A}}}
\newcommand{\blditB}{\textbf{\textit{B}}}
\newcommand{\blditC}{\textbf{\textit{C}}}
\newcommand{\blditP}{\textbf{\textit{P}}}
\newcommand{\blditQ}{\textbf{\textit{Q}}}
\newcommand{\bldI}{\textbf{I}}
\newcommand{\blditX}{\textbf{\textit{X}}}
\newcommand{\blditY}{\textbf{\textit{Y}}}
\newcommand{\blditZ}{\textbf{\textit{Z}}}
 
\newenvironment{theorem}[2][Theorem]{\begin{trivlist}
\item[\hskip \labelsep {\bfseries #1}\hskip \labelsep {\bfseries #2.}]}{\end{trivlist}}
\newenvironment{lemma}[2][Lemma]{\begin{trivlist}
\item[\hskip \labelsep {\bfseries #1}\hskip \labelsep {\bfseries #2.}]}{\end{trivlist}}
\newenvironment{exercise}[2][Exercise]{\begin{trivlist}
\item[\hskip \labelsep {\bfseries #1}\hskip \labelsep {\bfseries #2.}]}{\end{trivlist}}
\newenvironment{problem}[2][Problem]{\begin{trivlist}
\item[\hskip \labelsep {\bfseries #1}\hskip \labelsep {\bfseries #2.}]}{\end{trivlist}}
\newenvironment{question}[2][Question]{\begin{trivlist}
\item[\hskip \labelsep {\bfseries #1}\hskip \labelsep {\bfseries #2.}]}{\end{trivlist}}
\newenvironment{answer}[2][Answer]{\begin{trivlist}
\item[\hskip \labelsep {\bfseries #1}\hskip \labelsep {\bfseries #2.}]}{\end{trivlist}}

\begin{document}
 \renewcommand{\arraystretch}{1.3}

 
\title{Stat 8003, Homework 5}%replace X with the appropriate number
\author{Group G: \ \ \texttt{sample( c( "David" , "Andrew",  "Salam" ))}
\\ %replace with your name
} %if necessary, replace with your course title
 
\maketitle
 
 %%%%%%%% Question 1 %%%%%%%%%
 \begin{question}{5.1}  
 
 Consider a simulated dataset. Assume that the data $x_1, x_2, \cdots , x_n$ follows the following distribution:
 $$
 x_i \sim f(x_i) = \pi_0 f_0(x_i) + \pi_1 f_1(x_i)
 $$
 where $f_0(x_i) = 1(0 \leq x_i \leq 1)$ is the density function of the uniform and $f_1(x_i) = \beta (1 - x)^{\beta - 1} $ is the density function of $Beta(1, \beta)$. The group information can be treated as a
missing value and is denoted as $z_i$. Let $y_i = (x_i, z_i)$ be the complete data.	
 \begin{enumerate}[(a)]
\item Derive the complete likelihood function;
\item Use the EM algorithm to derive the estimator for $\pi_0$ and $\beta$;
\item Apply your method to the data set, estimate $\pi_0$ and $\beta$ and the calculate $fdr_i = P(Z_i = 0 \mid x_i)$. (This score is called the local fdr score.)
\item Classify $x_i$ to the first group if $fdr_i(x_i) > 0.5$. Compare your classification with the actual
group information, what is the total number of falsely classified data?
\end{enumerate}
\end{question} 


  \textbf{\color{TealBlue}\emph{Answer:} } 
    
  
\begin{enumerate}[(a)]  
\item First, the \emph{incomplete} likelihood function is given to be:
  $$
  L(\theta \ ; \blditX) = \prod_{i=1}^n \left(   \pi_0 1 + \pi_1 \beta (1 - x_i)^{\beta - 1}   \right)
  $$



Then the \emph{complete} likelihood function is:
$$\boxed{
L(\theta \ ; \blditY) = \prod_{i=1}^n \left( 1 (Z_i = 0) \ \pi_0  + 1( Z_i = 1) \ \pi_1 \ \beta (1 - x_i)^{\beta - 1} \right) }
$$ 
  
  An alternative way of writing this likelihood is:
  \begin{align*}
   f(x_i, z_i \mid \theta) = 
  \begin{cases}
  \pi_0 \; \; \; \; \; \; \; \; \; \; \; \; \; \; \; \; \; \; \; \; \; \; \; \; \; \text{if} \; Z_i = 0\\
  \pi_1 \ \beta (1 - x_i)^{\beta - 1}\; \; \; \; \; \text{if}  \; Z_i = 1\\
  \end{cases}
  \end{align*}
\item To get the estimates for $\pi_0$ and $\beta$, we first find the expected value of the \emph{log} of the \emph{complete likelihood} function with respect to $Z$ (the so called $Q$ function):

\begin{align*}
Q(\theta \mid \theta^t) &= \mathrm{E} \; \log (L(\theta \ ; \blditY) ) \\
&= \mathrm{E} \; \log \left( \prod_{i=1}^n \left( 1 (Z_i = 0)\ \pi_0   + 1( Z_i = 1)\ \pi_1  \ \beta (1 - x_i)^{\beta - 1} \right) \right)\\
&= \mathrm{E} \; \left[ \sum_{i=1}^n  \log \left( 1 (Z_i = 0)\ \pi_0   + 1( Z_i = 1)\ \pi_1 \ \beta (1 - x_i)^{\beta - 1} \right) \right] \\
\end{align*}
  
  The last expression in the brackets is either $\log(\pi_0)$ or $ \log(\pi_1\ \beta (1 - x_i)^{\beta - 1})$, depending on the outcome of $Z$. So
  
\begin{align*}
Q(\theta \mid \theta^t) &= \sum_{i=1}^n \left( \; \; \mathrm{E} \; 1 (Z_i = 0)\ \log(\pi_0) + \mathrm{E} \; 1 (Z_i = 1)\ \log(\pi_1\ \beta (1 - x_i)^{\beta - 1}) \; \; \right) \\
&= \sum_{i=1}^n \left( \; \; P(Z_i = 0 \mid x_i, \theta)\ \log(\pi_0) +  P(Z_i = 1 \mid x_i, \theta)\ \log(\pi_1\ \beta (1 - x_i)^{\beta - 1}) \; \; \right) \\
\end{align*}
  
  Where the last equality follows because the expectation of the indicator function of a \texttt{r.v.} is simply the probability of the corresponding event. \\
  \\
  These probabilities will be computed using Bayes rule and denoted by $T_{ij}^t$:
\begin{align*}  
T_{ij}^t &= P(Z_i = j \mid x_i, \theta) = \frac{P( x_i \mid Z_i = j) P(Z_i = j) } {\sum_{j=0}^1  P( x_i \mid Z_i = j) P(Z_i = j) }  \; \; \; \; \text{for} \; \; j = 0, 1 \\
\end{align*} 
Thus
$$
T_{i0}^t = \frac{\pi_0}{\pi_0 + \pi_1 \ \beta (1 - x_i)^{\beta - 1}} 
$$
\\
$$
T_{i1}^t = \frac{\pi_1 \ \beta (1 - x_i)^{\beta - 1}} { \pi_0 + \pi_1 \ \beta (1 - x_i)^{\beta - 1} }
$$
Rewriting the $Q$ function:
\begin{align*}
Q(\theta \mid \theta^t) &= \sum_{i=1}^n \left( \;  T_{i0}^t \log(\pi_0) + T_{i1}^t \log(\pi_1\ \beta (1 - x_i)^{\beta - 1}) \; \right) \\
&= \sum_{i=1}^n \left(\;  T_{i0}^t \log(\pi_0) + T_{i1}^t \log(1 - \pi_0) \ + T_{i1}^t \log (\beta (1 - x_i)^{\beta - 1} \;  \right) \\
\end{align*}

Now we need to maximize this function with respect to $\pi_0$ and $\beta$......

\begin{verbatim}
.....
.....
.....
.....


invoke maximizer extrodinaire...

should have started earlier...
\end{verbatim}


\end{enumerate}

\vspace{1000 mm}









\begin{verbatim}






























\end{verbatim}

\bigskip
\bigskip
 %%%%%%%% Question 2 %%%%%%%%%
 \begin{question}{5.2} (Continued from Problem 1.)  It is known that the local fdr score can be written as 
$$
fdr_i(x_i) = \frac{\pi_0 f_0(x_i)} {f(x_i)}
$$
where $f(x_i)$ is the marginal density of $x_i$. Assume that $\pi = 0.7$.
 \begin{enumerate}[(a)]
\item Estimate $f(x_i)$ by using the kernel density estimation with Gaussian kernel and Silverman's $h$;
\item Estimate the local $fdr$ score;
\item  Using the same rule as in 1(d), calculate the total number of falsely classified data;
\item Choose the bandwidth using the maximum likelihood cross validation, repeat problem (a-c),
what is the total number of falsely classified data?
\item Which method works the best in terms of having the smallest classification error?
\end{enumerate}
\end{question} 


  \textbf{\color{TealBlue}\emph{Answer:} } 
 
 
 

\end{document}

%Rewriting the $Q$ function:
%\begin{align*}
%Q(\theta \mid \theta^t) &= \sum_{i=1}^n \left( \frac{\pi_0 \ \log(\pi_0) \ + \ \pi_1 \ \beta (1 - x_i)^{\beta - 1} \log(\pi_1\ \beta (1 - x_i)^{\beta - 1})} {\pi_0 + \pi_1 \ \beta (1 - x_i)^{\beta - 1}} \right) \\
%&= \sum_{i=1}^n \left( \frac{\pi_0 \ \log(\pi_0) \ + \ \pi_1 \ \beta (1 - x_i)^{\beta - 1} \log(\pi_1\ \beta (1 - x_i)^{\beta - 1})} {\pi_0 + \pi_1 \ \beta (1 - x_i)^{\beta - 1}} \right) \\
%\end{align*}
%
%We can express this in terms of $\pi_0$ and $\beta$ only:
%\begin{align*}
%Q(\theta \mid \theta^t) &= \sum_{i=1}^n \left( \frac{\pi_0 \ \log(\pi_0) \ + \ (1 - \pi_0) \ \beta (1 - x_i)^{\beta - 1} \log((1 - \pi_0)\ \beta (1 - x_i)^{\beta - 1})} {\pi_0 + (1 - \pi_0) \ \beta (1 - x_i)^{\beta - 1}} \right) \\
%&= \sum_{i=1}^n \left( \frac{\pi_0 \ \log(\pi_0) \ + \ (1 - \pi_0) \ \beta (1 - x_i)^{\beta - 1} ( \log((1 - \pi_0)\ + \log (\beta (1 - x_i)^{\beta - 1}) )} {\pi_0 +  \ \beta (1 - x_i)^{\beta - 1} - \pi_0 \beta (1 - x_i)^{\beta - 1} }\right) \\
%\end{align*}
%\begin{align*}
%\sum_{i=1}^n \left( \frac{\pi_0 \ \log(\pi_0) \ + \ (1 - \pi_0) \ \beta (1 - x_i)^{\beta - 1} \log((1 - \pi_0)\ + \ (1 - \pi_0) \ \beta (1 - x_i)^{\beta - 1} \log (\beta (1 - x_i)^{\beta - 1}) )} {\pi_0 +  \ \beta (1 - x_i)^{\beta - 1} - \pi_0 \beta (1 - x_i)^{\beta - 1} }\right) 
%\end{align*}
%
